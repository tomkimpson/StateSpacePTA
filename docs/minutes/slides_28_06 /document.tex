% !TeX TS-program = xelatex

\documentclass[10pt]{beamer}

\usetheme[progressbar=frametitle]{metropolis}
\usepackage{appendixnumberbeamer}
\usepackage{pgfplots}
\usepackage{xspace}
\usepackage[normalem]{ulem}



\definecolor{links}{HTML}{2A1B81}

\hypersetup{colorlinks,linkcolor=,urlcolor=links}


\newcommand{\themename}{\textbf{\textsc{metropolis}}\xspace}
%\setsansfont[BoldFont={Fira Sans}]{Fira Sans Light}
%\setmonofont{Fira Mono}
%\usepackage[sfdefault]{Fira Sans}

\title{}
\subtitle{}

\author{}

\date{EE/GW meeting, June 29, 2023}

\begin{document}
	
	\maketitle
	
	\begin{frame}{}
		
		\begin{enumerate}
			\item IPTA conference
			\item Manuscript updates
			\item Identifiability analysis and frequency-space formulation from EE
		\end{enumerate}
	
\end{frame}



	\begin{frame}{IPTA conference}
	
	
	
	\begin{itemize}
		\item IPTA conference last week
		\item \sout{ Many results currently under embargo}
		\item \href{https://iopscience.iop.org/article/10.3847/2041-8213/acdac6}{The NANOGrav 15 yr Data Set: Evidence for a Gravitational-wave Background} 
		\item More papers to come c.f. single source. TK to sync with EE to discuss results and methods
	\end{itemize}
	
\end{frame}

	
	
	
	
	
	
	
		\begin{frame}{Manuscript updates}
		
		
		
		\begin{itemize}
			\item Plan for papers is as follows:
			\begin{enumerate}
				\item Single source. Earth terms + nested sampling (Bilby) 
				\item Single source. Pulsar terms, Earth terms + other likelihood inference methods (e.g. Bilby vs other nested sampling libs, MCMC, EM)
				\item Multiple sources (i.e. stochastic background, Hellings Downs)
			\end{enumerate}
			\item P1 manuscript currently being written up. Can be found at \href{https://github.com/tomkimpson/StateSpacePTA.jl/blob/main/docs/manuscript/paper.pdf}{github/StateSpacePTA.jl} 
			\item Majority of work for P1 is done - just a question of writing up
			\item Exception: comparison with existing approaches (e.g. \href{https://ui.adsabs.harvard.edu/abs/2014ApJ...794..141A/abstract}{NANOGrav 6 year}, \href{https://ui.adsabs.harvard.edu/abs/2019ApJ...880..116A/abstract}{NANOGrav 11 year}, \href{https://ui.adsabs.harvard.edu/abs/2023arXiv230103608A/abstract}{NANOGrav 12.5}, NANOGrav15). \alert{To discuss (next slide)}
			\item Completing P1 manuscript will be main focus over next few weeks, then back to a "research focus" re P2/P3
		\end{itemize}
		
	\end{frame}



	
			\begin{frame}{How to compare against existing methods?}
		
		
		Todo: comparison with existing approaches (e.g. \href{https://ui.adsabs.harvard.edu/abs/2014ApJ...794..141A/abstract}{NANOGrav 6 year}, \href{https://ui.adsabs.harvard.edu/abs/2019ApJ...880..116A/abstract}{NANOGrav 11 year}, \href{https://ui.adsabs.harvard.edu/abs/2023arXiv230103608A/abstract}{NANOGrav 12.5}, NANOGrav15). 
		
		Existing methods:
		\begin{enumerate}
			\item Take some timing residuals ``observations" $\delta t$, (i.e. TOAs  - expected TOAs from timing model $M \epsilon$)
			\item Define a GW model for timing residuals $s(\theta_{\rm gw})$. 
			\item Define a noise model $n(\theta_{\rm n})$
			\item Define a likelihood $\mathcal{L}(\delta t | \epsilon,\theta_{\rm gw},\theta_{\rm n} )$
			\item Two approaches
			\begin{enumerate}
				\item Frequentist. Define an F-statistic from $\mathcal{L}$ (now disfavoured?)
				\item Bayesian. MCMC over parameters
			\end{enumerate}
			
		\end{enumerate}
		
 How to compare? Do we need to compare for P1? 


	\end{frame}
	
	
	
			\begin{frame}{Identifiability and frequency-space}
		
		
		\begin{itemize}
			\item Rob + Bill presented frequency-space formulation of problem 
			\item Interesting/useful for identifiability analysis + potentially as an alternative inference/detection method
		\end{itemize}
	

		
	\end{frame}
	
	
	\appendix
	
%	
%	
%	\begin{frame}
%		
%		
%		
%		\begin{itemize}
%			\item CURN vs HD \url{https://arxiv.org/abs/2110.13184}
%			\item Model misspecification? \url{https://iopscience.iop.org/article/10.3847/2041-8213/ac17f4}
%		\end{itemize}
%	\end{frame}
%	
	
	
	
	
\end{document}