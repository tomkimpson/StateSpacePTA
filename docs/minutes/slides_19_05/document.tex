% !TeX TS-program = xelatex

\documentclass[10pt]{beamer}

\usetheme[progressbar=frametitle]{metropolis}
\usepackage{appendixnumberbeamer}
\usepackage{pgfplots}
\usepackage{xspace}

\newcommand{\themename}{\textbf{\textsc{metropolis}}\xspace}
%\setsansfont[BoldFont={Fira Sans}]{Fira Sans Light}
%\setmonofont{Fira Mono}
%\usepackage[sfdefault]{Fira Sans}

\title{}
\subtitle{}

\author{}

\date{EE/GW meeting, May 19, 2023}

\begin{document}
	
	\maketitle
	
	\begin{frame}{}
		
		Last time: Problems at small $h$, nasty likelihood curves
	
	All likelihood curves for $\bar{\theta}_{\rm GW}$:

\includegraphics[width=0.9\textwidth,height=0.7\textwidth]{images/all_likelihood_curves}


\end{frame}




\begin{frame}{Drop PSR terms from model?}
	\begin{equation}
		f_{\rm measured} = f_{\rm emitted} \, g(\tau; \bar{\theta})
		\label{eq:measureent}
	\end{equation}
	with
	\begin{align}
		g(\tau; \bar{\theta}) 
		& = 1 - \frac{1}{2} \frac{ H_{ij}q^i q^j }{(1 + \bar{n}\cdot \bar{q}) } \left[ \cos(-\Omega \tau +\Phi_0) - \cos(-\Omega \tau +\Phi_0 + \Omega (1 + \bar{n}\cdot \bar{q})  d) \right]
	\end{align}
	
	Generate the fake data using the full expression for $g(\tau; \bar{\theta})$, but use $g'(\tau; \bar{\theta})$ in the Kalman measurement model
	
	\begin{align}
		g'(\tau; \bar{\theta}) 
		& = 1 - \frac{1}{2} \frac{ H_{ij}q^i q^j }{(1 + \bar{n}\cdot \bar{q}) } \left[ \cos(-\Omega \tau +\Phi_0) \right]
	\end{align}
\end{frame}


\begin{frame}{Drop PSR terms from model?}
	
	All likelihood curves for $\bar{\theta}_{\rm GW}$:
	\includegraphics[width=0.9\textwidth,height=0.7\textwidth]{images/all_likelihood_curves_psr}
	
\end{frame}



\begin{frame}
	Full NS search for \alert{all parameters} \newline 
	$\bar{\theta} = [\bar{\theta}_{\rm GW}, \, f, \, (t=0) \, \dot{f} \, (t=0) ]$ \newline 
	with $h = 10^{-10}$, $\sigma_m = 10^{-11}$ \newline 
	Note: pulsar distance is no longer part of model. $\gamma$ doesn't matter
	

\end{frame}


\begin{frame}
		\includegraphics[width=0.9\textwidth,height=0.9\textwidth]{images/canonical_mixed_result}
\end{frame}



\begin{frame}
	\includegraphics[width=0.9\textwidth,height=0.9\textwidth]{images/canonical_mixed_f_result}
\end{frame}



\begin{frame}
	Next steps:
	\begin{itemize}
		\item Bayes factors vs. strain (currently running)
		\item Process noise $\sigma_p$
		\item Looser priors on $f$, $\dot{f}$? Do we need to?
		\item Can we add PSR terms back in? Do we need to?
	\end{itemize}
\end{frame}


\begin{frame}[standout]
	Fin.
\end{frame}

\appendix

\begin{frame}[fragile]{Backup slides}
	Sometimes, it is useful to add slides at the end of your presentation to
	refer to during audience questions.
	
	The best way to do this is to include the \verb|appendixnumberbeamer|
	package in your preamble and call \verb|\appendix| before your backup slides.
	
	\themename will automatically turn off slide numbering and progress bars for
	slides in the appendix.
\end{frame}


\begin{frame}[fragile]{Where did $\sigma_m$ value come from?}
The measurement noise is

$$ \sigma_m = f \frac{\sigma_{\rm TOA}}{\rm cadence}$$

So for a MSP ( $f \sim 100 Hz$) observed with a weekly cadence and $\sigma_{\rm TOA} \sim 1 \mu$s:


$$\sigma_m \sim 1.6 \times 10^{-10}$$

The very best pulsars might have $\sigma_{\rm TOA} \sim 10$ ns:

$$\sigma_m \sim 1.6 \times 10^{-12}$$
\end{frame}







\begin{frame}{$\gamma$ likelihood}
	\includegraphics[width=0.8\textwidth,height=0.8\textwidth]{images/likelihood_gamma_1}
\end{frame}


	\begin{frame}{}
	
	\begin{itemize}
		\item Why? Very nasty likelihood curves
	\end{itemize}
	
	
	$$  p(\theta | d) \propto p(d | \theta) \pi({\theta})$$
	$$ p(\theta | d) \propto \mathcal{L}(\theta) \pi({\theta})$$
	
		
	For uniform prior, posterior shape $\sim$ likelihood shape. 


\end{frame}



\begin{frame}{}
	
	Example: $\mathcal{L}(\delta)$
	
\includegraphics[width=0.7\textwidth,height=0.6\textwidth]{images/likelihood_delta_1}
	
	
\end{frame}

\begin{frame}{}
	
	Example: $\mathcal{L}(\delta)$, zoomed 1
	
	\includegraphics[width=0.7\textwidth,height=0.6\textwidth]{images/likelihood_delta_2}
	
	
\end{frame}


\begin{frame}{}
	
	Example: $\mathcal{L}(\delta)$, zoomed 2
	
	\includegraphics[width=0.7\textwidth,height=0.6\textwidth]{images/likelihood_delta_3}
	
	
\end{frame}


\begin{frame}{}
	
	Example: $\mathcal{L}(\omega)$
	
	\includegraphics[width=0.7\textwidth,height=0.6\textwidth]{images/likelihood_omega_1}

	
\end{frame}


\begin{frame}{}
	
	Example: $\mathcal{L}(\omega)$, zoomed 1
	
	\includegraphics[width=0.7\textwidth,height=0.6\textwidth]{images/likelihood_omega_2}
	
	
\end{frame}


\begin{frame}{}
	
	Example: $\mathcal{L}(\omega)$, zoomed 2
	
	\includegraphics[width=0.7\textwidth,height=0.6\textwidth]{images/likelihood_omega_3}
	
	
\end{frame}






\begin{frame}{}
	
		\begin{itemize}
		\item \alert{Problem} = overly narrow,biased posteriors
		\item Cause: noisy, likelihoods which are "locally Gaussian". Sampling gets stuck in a local optima and infers posteriors widths based on that optima. 
%		\item Require $n_{live} > \frac{\rm prior \, volume}{\rm mode \,  footprint} = \frac{X}{\sigma^2}$
%		\begin{itemize}
%			\item For $\delta$ require $n_{live} > \frac{2 \pi}{10^{-5}} \sim 10^5 $
%			\item For $\omega$ require $n_{live} > \frac{10^4}{10^{-11}} \sim 10^7 $ 
%			\item Currently $n_{live} = 500$ parallelized on 32 cores $\sim 700$ minutes
%			\item Runtime scales $\mathcal{O}(n_{live}) \rightarrow 8000 $ hours for required $\delta$ sampling (just under 1 year!)
%
%		\end{itemize}
	\end{itemize}
	
	
	
\end{frame}


\begin{frame}{}
	Potential solutions:
	
	\begin{enumerate}
		\item Better settings for sampler? E.g. more live points (how many?), massively parallel, optimized likelihood evaluations,...
		\item Maximum likelihood / loss optimisation? 
		\item Drop PSR terms from model?
		\item MCMC at higher temperatures (trades off evidence..., less efficient	)
		\item other...?
	\end{enumerate}
	
	
\end{frame}





\begin{frame}{Option 2: Drop PSR terms from model?}
	
	
	\includegraphics[width=0.9\textwidth,height=0.7\textwidth]{images/canonical_mixed_result}
	
	OK, but some biases. c.f. location of the maxima on the previous slide
\end{frame}



\begin{frame}{}
	
	Lets proceed using this `Earth terms' model for now. 
	
	
   So far we have been considering inference of only the GW terms $\bar{\theta}_{\rm GW}$. It is important to be able to infer the pulsar terms $\bar{\theta}_{\rm PSR}$.
   
   
   Lets try and include $f \, (t=0)$ and $\dot{f} \, (t=0)$ in our nested sampling.
   
   We have a uniform prior of $\pm 1 \%$ about the true value for these two parameters 
	
	
	

	

\end{frame}

\begin{frame}{Earth terms model $\bar{\theta}_{\rm GW}$ and $f \, (t=0)$, $\dot{f} \, (t=0)$}
	\includegraphics[width=0.9\textwidth,height=0.7\textwidth]{images/canonical_mixed_result}
\end{frame}


%	\begin{frame}{}
	%
	%Numerical weather and climate modelling 
	%
	%        \begin{itemize}
		%	        \item Reduced precision floats + stochastic rounding
		%	        \item Automatic Differentiation
		%        \end{itemize}
	%    
	%    \alert{Open call for collaboration!}
	%    
	%    
	%	\end{frame}



	
\end{document}