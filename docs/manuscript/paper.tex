% mnras_guide.tex
%
% MNRAS LaTeX user guide
%
% v3.1 released 11 June 2020
%
% v3.0 released 22 May 2015
% (version numbers match those of mnras.cls)
%
% Copyright (C) Royal Astronomical Society 2015
% Authors:
% Keith T. Smith (Royal Astronomical Society)

% Change log
%
% v3.0   September 2013 - May 2015
%    First version: complete rewrite of the user guide
%    Basic structure taken from mnras_template.tex by the same author

%%%%%%%%%%%%%%%%%%%%%%%%%%%%%%%%%%%%%%%%%%%%%%%%%%
% Basic setup. Most papers should leave these options alone.
\documentclass[fleqn,usenatbib,useAMS]{mnras}

%%%%% AUTHORS - PLACE YOUR OWN PACKAGES HERE %%%%%

% Only include extra packages if you really need them. Common packages are:
\usepackage{graphicx}	% Including figure files
\usepackage{amsmath}	% Advanced maths commands
\usepackage{amssymb}	% Extra maths symbols
\usepackage{multicol}        % Multi-column entries in tables
\usepackage{bm}		% Bold maths symbols, including upright Greek
\usepackage{pdflscape}	% Landscape pages

%%%%%%%%%%%%%%%%%%%%%%%%%%%%%%%%%%%%%%%%%%%%%%%%%%

%%%%%% AUTHORS - PLACE YOUR OWN MACROS HERE %%%%%%

% Please keep new commands to a minimum, and use \newcommand not \def to avoid
% overwriting existing commands. Example:
%\newcommand{\pcm}{\,cm$^{-2}$}	% per cm-squared
\newcommand{\kms}{\,km\,s$^{-1}$} % kilometres per second
\newcommand{\bibtex}{\textsc{Bib}\!\TeX} % bibtex. Not quite the correct typesetting, but close enough

%%%%%%%%%%%%%%%%%%%%%%%%%%%%%%%%%%%%%%%%%%%%%%%%%%


% Use vector fonts, so it zooms properly in on-screen viewing software
% Don't change these lines unless you know what you are doing
\usepackage[T1]{fontenc}
\usepackage{ae,aecompl}

% MNRAS is set in Times font. If you don't have this installed (most LaTeX
% installations will be fine) or prefer the old Computer Modern fonts, comment
% out the following line
\usepackage{newtxtext,newtxmath}
% Depending on your LaTeX fonts installation, you might get better results with one of these:
%\usepackage{mathptmx}
%\usepackage{txfonts}

%%%%%%%%%%%%%%%%%%% TITLE PAGE %%%%%%%%%%%%%%%%%%%

% Title of the paper, and the short title which is used in the headers.
% Keep the title short and informative.
	\title[Kalman PTA]{State-space PTA}

% The list of authors, and the short list which is used in the headers.
% If you need two or more lines of authors, add an extra line using \newauthor
\author[Kimpson]{Kimpson$^{1}$, O'Leary, Melatos, Evans, others, etc. %
\thanks{Contact e-mail: \href{mailto:mn@ras.ac.uk}{mn@ras.ac.uk}}%
\thanks{Present address: Science magazine, AAAS Science International, \mbox{82-88}~Hills Road, Cambridge CB2~1LQ, UK}%
\\
% List of institutions
$^{1}$Royal Astronomical Society, Burlington House, Piccadilly, London W1J 0BQ, UK}

% These dates will be filled out by the publisher
\date{Last updated 2020 June 10; in original form 2013 September 5}

% Enter the current year, for the copyright statements etc.
\pubyear{2020}

% Don't change these lines
\begin{document}
\label{firstpage}
\pagerange{\pageref{firstpage}--\pageref{lastpage}}
\maketitle

% Abstract of the paper
\begin{abstract}
This is an abstract
\end{abstract}

% Select between one and six entries from the list of approved keywords.
% Don't make up new ones.
\begin{keywords}
editorials, notices -- miscellaneous
\end{keywords}

%%%%%%%%%%%%%%%%%%%%%%%%%%%%%%%%%%%%%%%%%%%%%%%%%%

%%%%%%%%%%%%%%%%% BODY OF PAPER %%%%%%%%%%%%%%%%%%

% The MNRAS class isn't designed to include a table of contents, but for this document one is useful.
% I therefore have to do some kludging to make it work without masses of blank space.
\begingroup
\let\clearpage\relax
%\tableofcontents
\endgroup
\newpage

\section{Introduction}


\begin{itemize}
	\item Introduce PTAs generally.
	\item Types of astrophysical source to be detected with PTAs.
	\item Why we want to do parameter estimation
	\item Advantages of doing this with state-space approach
\end{itemize}




\section{Model}
The intrinsic pulsar frequency $f$ evolves as (See Vargas and Melatos)
\begin{equation}
	\frac{df}{dt} = -\gamma	 [f - f_{\rm EM} (t)] + \dot{f}_{\rm EM} + \xi(t)
	\label{eq:frequency_evolution}
\end{equation}
where $f_{\rm EM}$ is the solution of the electromagnetic spindown equation, $\gamma$ a proporionality constant, and $\xi(t)$ a white noise process that satisfies,
\begin{equation}
	\langle \xi(t) \xi(t') \rangle = \sigma^2 \delta(t - t')
\end{equation}
Over the timescales that we are interested in, we can express the EM spindown simply as
\begin{equation}
	f_{\rm EM} (t) = f_{\rm EM}(0) + \dot{f}_{\rm EM} (0)t
\end{equation}  
Completely, the frequency evolution is then
\begin{equation}
	\frac{df}{dt} = -\gamma	 [f - f_{\rm EM}(0) - \dot{f}_{\rm EM} (0)t] + \dot{f}_{\rm EM} (0) + \xi(t)
	\label{eq:frequency_evolution_expanded}
\end{equation}
This intrinsic frequency can be related to a measured frequency as 
\begin{equation}
	f_M = f g(\bar{\theta},t) + N_M
\end{equation}
where $g(\theta,t)$ ("measurement function")is a function of some parameters, $\bar{\theta}$ and time $t$, whilst $N_M$ is a Gaussian measurement noise that satisfies 
\begin{equation}
	\langle N_M(t) N_M(t') \rangle = \Sigma^2 \delta(t - t')
\end{equation}
Explicitly, it can be shown that the measurement function is,
\begin{equation} \label{eq:final}
	g(\bar{\theta},t) = 1 - \frac{1}{2} \frac{h_{ij}(t) q(t)^i q(t)^j}{1 + \bar{n} \cdot \bar{q}(t)}[1 - e^{i \Omega(1+\bar{n} \cdot \bar{q}(t)) d}]
\end{equation}
There are a few terms to unpack and define in this equation. $q(t)$ is the vector that connects the Earth and the pulsar, $n$ the vector that connects the Earth and the GW source, $\Omega$ the (constant) angular frequency of the GW and $d$ the Earth-pulsar distance. $h_{ij}(t)$ is the metric perturbation due to the GW $= g_{ij} - \eta_{ij}$. It is given by
\begin{equation}
h_{ij}(t) = H_{ij} e^{(i(\Omega (\bar{n} \cdot \bar{q} - t) + \Phi_0))}
\end{equation}
for phase offset (GW phase at Earth when $t=0$) $\Phi_0$. The amplitude tensor  $H_{ij}$ is
\begin{eqnarray}
	H_{ij} = h_+ e_{ij}^+(\bar{n},\psi) + h_{\times} e_{ij}^{\times}(\bar{n},\psi)
\end{eqnarray}
where $h_{+,\times}$ are the constant amplitudes of the gravitational plane wave, and $e_{ij}^{+, \times}(\bar{n}, \psi)$ are the polarisation tensors which are uniquely defined by the principal axes of the GW. $\psi$ is the polarisation angle of the GW. \newline 


\noindent Going forward, for now we will take $q(t) = q$ i..e. the pulsar locations are constant with respect to the Earth. This may have already been "done" during the barycentreing when pulsar TOAs are generated, in which case $q$ is the vector from the SSB to the pulsar  \newline 


\noindent Lets review and categorise all the parameters:
\begin{equation}
	\bar{\theta} =  \bar{\theta}_{\rm PSR} + \bar{\theta}_{\rm GW} + \bar{\theta}_{\rm noise}
\end{equation}
\begin{equation}
	\bar{\theta}_{\rm PSR} = [\gamma, f_{\rm EM}(0), \dot{f}_{\rm EM}(0),d]
\end{equation}
\begin{equation}
	\bar{\theta}_{\rm GW} = [h_{+}, h_{\times}, \delta, \alpha, \psi, \Omega, \Phi_0]
\end{equation}
\begin{equation}
	\bar{\theta}_{\rm noise} = [\sigma, \Sigma]
\end{equation}
We can also express the measurement equation generally as,
\begin{equation} 
	g(\bar{\theta},t) = 1 -A \cos(\Omega(n\cdot q - t) + \Phi_0)
\end{equation}
where 
\begin{equation}
	A =  \frac{1}{2} \frac{H_{ij} q^i q^j}{1 + \bar{n} \cdot \bar{q}}[1 - e^{i \Omega(1+\bar{n} \cdot \bar{q}) d}]
\end{equation}


\subsection{References}
\label{sec:ref_list}





\bibliographystyle{mnras}
\bibliography{example} % if your bibtex file is called example.bib





%%%%%%%%%%%%%%%%%%%%%%%%%%%%%%%%%%%%%%%%%%%%%%%%%%


% Don't change these lines
\bsp	% typesetting comment
\label{lastpage}
\end{document}

% End of mnras_guide.tex
